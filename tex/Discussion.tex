\chapter{Conclusion}
In this thesis, we set out to investigate pion photoproduction in a nuclear model with explicit mesons. To do this, we first introduced the model in chapter \ref{Decsofmodel} and considered how a general pion-nucleon system could be investigated. We then focused on the dressing of the proton specifically and introduced a method of solving the wave function numerically. Afterwards, we exploited the model's generality to consider different form factors and how this affected the solutions. We also considered a different operator type related to the operator found in effective field theories. This model turned out to be similar but much more numerically intensive. 

In chapter \ref{sec:PionPhotoproduction}, we began describing how pion photoproduction could be considered in a nuclear model with explicit pions. As a first approach, we used a dipole approximation which was valid only very close to the threshold. In the case of neutral pion photoproduction off protons, the model turned out to describe the total cross-section very accurately when compared to experimental data. The parameters needed for the model could also be used to describe the behaviour of the differential cross-section. In the case of neutral pion photoproduction on neutrons, experimental data is needed before the model's parameters can be extracted. A theoretical prediction was made assuming similar parameters to the case of neutral pions off protons. 

In the case of charged pion photoproduction off nucleons, expressions for the total cross section and the charge density were found. The model has some problems accurately describing the behaviour of the total cross-section of charged pions, and perhaps the one-pion approximation is insufficient, and further work is needed. This would include a theoretical investigation of the two-pion approximation as well as a more suitable numerical approach. 

