\chapter{Conclusion}
In this thesis, we set out to investigate pion photoproduction in a nuclear model with explicit mesons. To do this, we first introduced the model in chapter \ref{Decsofmodel} and considered how a general pion-nucleon system could be investigated. We then focused on the dressing of the proton specifically and introduced a method of solving the wave function numerically. We then exploited the model's generality to consider different form factors and how this affected the solutions. We also considered a different operator type related to the operator found in effective field theories. This model turned out to be similar but much more numerically intensive. 
In chapter \ref{sec:PionPhotoproduction} we began describing how pion photoproduction could be considered in a nuclear model with explicit pions. The dipole approximation was valid only very close to the threshold. In the case of neutral pion photoproduction off protons, the model turned out to very accurately describe the total cross-section. The parameters needed for the model could also be used to describe the behaviour of the differential cross-section. However, some problems still arise, and further work is needed. The nuclear model with explicit pions turns out to be adequate for describing a phenomenon within the regime of low-energy physics, and this motives the investigation of similar phenomena.
