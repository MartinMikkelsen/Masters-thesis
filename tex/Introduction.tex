e\chapter{Introduction}
Nuclear physics covers and expands different ideas from other areas of physics. These include but are not limited to low- and high-energy physics, few- and many-body dynamics, and classical and quantum statistical mechanics. Two concepts are needed when discussing nuclear physics: the nucleon and mesons. Protons and neutrons have very similar properties, and the unifying term nucleon refers to a mass of around 1 GeV. Nuclei consist of nucleons and are held together by the nuclear forces--by exchanging mediating quanta called mesons. This is similar to how the photon exchange generates the electromagnetic force. There are many different mesons, but the lightest mesons are called the pions ($\pi^-,\pi^0,\pi^+$) with a mass of about one-seventh of the nucleon leaving us in the MeV range. This energy scale also defines the regime known as low-energy physics, where the nucleus can be considered non-relativistic, and the mesons are virtual particles hidden in nucleon-nucleon interactions.
Generally speaking, within the domain of low-energy nuclear physics, the nucleus appears as a self-bound many-body nucleonic system with intrinsic degrees of freedom. These systems are mesoscopic, along with atoms, molecules, micro- and nano-devices of condensed matter systems, and quantum computers. This means they are sufficiently large to have statistical regularities yet also small enough to study individual quantum states.

 Increasing the energy will reveal the intermediate region of nuclear physics. Here relativistic effects become more important, and the meson and nucleon excitations become explicit. This energy scale is loosely characterized by an energy scale of a few GeV. At even higher energies and higher momentum transfer, dissecting the constituent of the nucleons and mesons is possible. These are known as quarks and gluons, and now the energy scale is in the TeV range. Generally speaking, the field of nuclear physics covers an energy range from KeV to TeV and the two different regimes of relativistic effects. 
\section{Strong interactions}
Using the uncertainty relation one can estimate the range of the nuclear forces. If we assume the interaction is mediated by a quanta being emitted by one particle and absorbed by another particle we gain insight into two properties of the strong nuclear force. Firstly, how the pion exchange as a mechanicsm for the nucleon interaction and how the range of the strong nuclear force is charachterized by the Compton wavelength of the lightest possible mediator. In the case of the neutral pion is this approximately 1.46 fm. 
\section{Outline of Thesis}
This thesis is organised as follows. We will introduce the model in chapter \ref{Decsofmodel}. We will go through how the nuclear model with explicit mesons will be constructed generally. Then we consider the case where the mesons are pions and look at how we can determine an equation of motion for the pion-nucleon system. We then focus specifically on the one-pion approximation, which is the most straightforward appearance of the nuclear model with explicit pions off protons. This is closely related to how we formulate the dressing of the proton and arrive at an equation of motion that is to be solved. This is the subject of section \ref{sec:numericalconsiderations} where we explore the flexibility of this model. Specifically, this means we test different form factors and do a relativistic expansion to explore how this affects the solutions found in the previous section. Finally, we test how changing the operator to an operator found in effective field theory affects the system. In section \ref{sec:PionPhotoproduction} we explore pion photoproduction using the model with explicit pions and how this emerges naturally as a photodisintegration process. We consider the four possible reactions: two off the proton and two off the neutron. To calculate the matrix elements, we make a dipole approximation in section \ref{sec:dipoleapprox}, and an exact approach follows this in section \ref{sec:exact}. We do fits to extract the parameters for the model and compare to experimental data. Some parts of the thesis made it into an article \cite{ThresholdPion}.
