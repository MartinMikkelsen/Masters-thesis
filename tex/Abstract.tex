% ~~~~~~~~~~~~
% The colophon
% ~~~~~~~~~~~~

% Get font-info
%\makeatletter
\edef\fontandleading{\@memptsize.0/\the\baselineskip}
\makeatother
\thispagestyle{empty}
% Push to bottom of page and locally set indents
\strut\vfill
{
	\setlength{\parindent}{0pt}
	\addtolength{\parskip}{.6em}
	
	\begin{center}
		\bfseries\sffamily Colophon
	\end{center}
	
	\small
	
	\textsl{\projecttitle}
	
	\smallskip
	
	Master's thesis by Martin Mikkelsen
	
	The project is supervised by Dmitri Fedorov
	
	Typeset by the author using \LaTeX{} and the \textsf{memoir} document class. Figures are made with the using the Seaborn package in Python and the Tikz package.
	
	Code for document, figures and numerical calculations can be found on \url{https://github.com/MartinMikkelsen}
	
	Printed at Aarhus University
}
\frontmatter
\section*{Abstract}
\addcontentsline{toc}{chapter}{Abstract}
%\thispagestyle{empty}

This thesis investigates pion photoproduction off nucleons near the threshold using a nuclear model with explicit pions. In this model, the nucleons do not interact through a potential but emit and absorb mesons which are treated explicitly, and we limit the model to the one meson approximation. We focus on the case where the mesons are pions and calculate the total cross-section of pion photoproduction near the threshold. Specifically, we find the set of parameters for which the model quantitatively can describe the total cross section near the threshold. 

We introduce the nuclear model with explicit mesons and describe the advantages of using this model within the regime of few-body, low-energy nuclear physics. We then consider the specific case where the mesons are pions and consider a pion-nucleus system where we introduce a phenomenological form factor which depends on a strength parameter and a range parameter. We then consider a numerical approach to solving the Schrödinger equation describing the pion-nucleon system and evaluate how changing the phenomenological form factor will affect the solutions. We also discuss a relativistic expansion of the kinetic operators to deduce the importance of relativistic effects on the pion-nucleon system. We find that the relativistic effects are negligible for most sets of parameters. To further test the model, we introduce a new operator which is inspired by an effective field theory. We find that the operator is compatible with the nuclear model but is also numerically intensive, and perhaps another numerical approach is more suitable for this operator.

We then focus on pion photoproduction and how this can be described within the framework of the nuclear model. As a first approach, we consider a dipole approximation and calculate the total cross-section for the photoproduction of charged pions off protons. We find that the dipole approximation is only valid very close to the threshold. We then consider the central challenge of this thesis: a general expression for the differential cross-section and the total cross-section near the threshold. We compute these expressions for all four pion photoproduction processes. We conclude that the model is able to describe the experimental cross-section for neutral pions off protons quantitatively, and we present the sets of parameters. At the time of writing, no experimental data exists for neutral pion photoproduction off neutrons near the threshold, but a theoretical prediction was made. In the case of charged pion photoproduction off nucleons, expressions for the total cross section and the charge density were found. The model has some problems accurately describing the behaviour of the total cross-section of charged pions, and perhaps the one-pion approximation is insufficient, and further work is needed.

\newpage
\section*{Resumé}
\addcontentsline{toc}{chapter}{Resumé}
Dette speciale undersøger pion fotoproduktion på nukleoner tæt på tærsklen gennem en kernemodel hvor mesonerne bliver behandlet eksplicit. I denne model vekselvirker nukleonerne ikke gennem et potentiale, men udsender og absorberer mesoner, som er håndteret eksplicit, og vi begrænser modellen til udelukkende at betragte en enkeltmesonapproksimation. Vi fokuserer på det tilfælde, hvor mesonerne er pioner og beregner det totale tværsnit af pion fotoproduktion tæt på tærsklen. Specifikt bestemmer vi de parametre, for hvilke modellen kvantitativt kan beskrive det totale tværsnit tæt på tærsklen. 

Vi introducerer kernemodellen med eksplicitte mesoner og beskriver de fordele denne model har i området af fålegeme, lavenergi kernefysik. Vi betragter det specifikke tilfælde, hvor mesonerne er pioner og betragter et pion-nukleon system, hvor vi introducerer en  fænomenologisk formfaktor, som afhænger af en styrkeparameter og en afstandsparameter. Derefter undersøger vi en numerisk fremgangsmåde til at løse Schödingerligningen, som beskriver pion-nukleon systemet og evaluerer hvordan  ændringer i formfaktoren vil påvirke løsningerne. Vi diskuterer, også en relativistisk udvidelse af operatoren for kinetisk energi og udleder vigtigheden af relativistiske effekter på pion-nukleon systemet. Vi finder, at relativistiske effekter er ubetydelige for de fleste sæt af parametre. For at teste modellen yderligere introducerer vi en ny operator, som er inspireret af en effektiv feltteori. Modellen er kompatibel med denne operator, men denne er tungere numerisk, og dette indikerer, at en anden numerisk metode kan være bedre egnet til denne operator.

Derefter fokuserer vi på pion fotoproduktion, og hvordan dette kan blive beskrevet indenfor kernemodellens teoretiske ramme. Som et første mål betragter vi en dipolapproksimation og beregner det totale tværsnit for ladet pion fotoproduktion fra protoner. Vi kommer frem til at dipolapproksimationen udelukkende gælder meget tæt på tærsklen. Derefter betragter vi hovedudfordringen i dette speciale: et generelt udtryk for differentialtværsnittet og for det totale tværsnit tæt på tærsklen. Vi udleder alle udtrykkene for de fire pion fotoproduktionsprocesser. Vi konkluderer, at modellen kvantitativt kan beskrive eksperimentelt data for fotoproduktionen af neutrale pioner fra protoner, og vi præsenterer flere sæt af parametre. I skrivende stund er der ikke eksperimentelt data for fotoproduktionen af neutrale pioner fra neutroner tæt på tærsklen, men vi præsenterer en teoretisk model. I tilfældet med ladede pioner finder vi frem til udtryk for det totale tværsnit og ladningstætheden. Undersøgelsen viser også, at modellen har nogle problemer, når det gælder beskrivelsen af tværsnittet af ladede pioner, hvilket indikerer at enkeltpionapproksimationen kræver yderligere overvejelser.

\newpage
\section*{Preface}
\addcontentsline{toc}{chapter}{Preface}
\thispagestyle{empty}
This thesis is the culmination of a year's work at the Department of Physics and Astronomy at Aarhus University under the supervision of Dmitri Fedorov. The objective was to test if the nuclear model with explicit pions could describe pion photoproduction off nucleons near the threshold.
I would like to thank my supervisor Dmitri Fedorov for introducing me to the topic and the discussions we had along the way. I have learned a great deal of new theoretical physics combined with numerical implementations. 

I would like to thank my friends and office colleagues, Magnus Linnet Madsen, Daniel Holleufer, Anton Lautrup, Ajanthan Ketheeswaran, Freja Nielsen, Lukas M. Wick and Sebastian Yde Madsen for their support and company throughout my time at Aarhus University. A special thanks to Magnus, Daniel and Ajanthan for revising the thesis.
The cover image shows a green nucleon interacting with a red particle. The image was made using an artificial intelligence program that creates images from textual descriptions called Midjourney. The figures in the margin will be used to supplement the text, while the figures in the text generally are the results.
\vspace{1cm}
\clearpage
\thispagestyle{empty}\mbox{}
\clearpage
